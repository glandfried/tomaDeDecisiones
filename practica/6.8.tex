\documentclass[a4paper,10pt,spanish]{article}

\usepackage[font={small}]{caption}
\input{../tex/encabezado.tex}
\input{../tex/tikzlibrarybayesnet.code.tex}

\usepackage[utf8]{inputenc}
\usepackage{fullpage}
\usepackage[hidelinks]{hyperref} % para inlcuir links dentro del texto

\makeatletter
\def\@xfootnote[#1]{%
  \protected@xdef\@thefnmark{#1}%
  \@footnotemark\@footnotetext}
\makeatother

%opening
\title{Ejercicio 6.8}
\author{Gustavo Landfried}
\date{}

\begin{document}

\maketitle

 \begin{figure}[H]     
    \centering \small
     \includegraphics[page=10,width=0.6\textwidth]{figures/forkingPath}
    
    \scriptsize Caminos totales (claros) y caminos posibles (oscuros) dada hip\'otesis \includegraphics[page=5,width=0.05\textwidth]{figures/forkingPath} y observaci\'on \includegraphics[page=1,width=0.043\textwidth]{figures/forkingPath}
    \caption{}
    \label{caminos}
\end{figure}

Tenemos en total 5 hip\'otesis $\{ \includegraphics[page=2,width=0.05\textwidth]{figures/forkingPath},\includegraphics[page=3,width=0.05\textwidth]{figures/forkingPath},\includegraphics[page=4,width=0.05\textwidth]{figures/forkingPath},\includegraphics[page=5,width=0.05\textwidth]{figures/forkingPath},\includegraphics[page=6,width=0.05\textwidth]{figures/forkingPath} \}$, 5 cartucheras posibles.
La cantidad de caminos que hacen verdadera a cada una de las hip\'otesis depende de la observaci\'on.
En la figura~\ref{caminos} vemos los caminos totales y posibles para una combinaci\'on de hip\'otesis \includegraphics[page=5,width=0.05\textwidth]{figures/forkingPath} y observaciones \includegraphics[page=1,width=0.043\textwidth]{figures/forkingPath}.
Cada combinaci\'on de hip\'otesis-observaci\'on tendr\'a una cantidad distinta de caminos posibles.
 

  
 \begin{table}[H]
 \tiny 
\begin{tabular}{cccccc}
\ \ \ Creencia \ \ \ & Caminos que conducen   \includegraphics[page=1,width=0.045\textwidth]{figures/forkingPath}  
& Verosimilitud & Priori & Posteriori $\propto$ & Posteriori
\\ \cline{1-6} \\[-0.2cm]

  \includegraphics[page=2,width=0.05\textwidth]{figures/forkingPath} &  $0 \times 4 \times 0 = 0$ 
 & $\frac{0 \times 4 \times 0 }{4 \times 4 \times 4 } = \frac{0}{64} $  & $1/5$ & $\frac{0}{64}\frac{1}{5}$ & $\frac{0}{3+8+9} =  0.00 $ 
 \\[2pt]
  \includegraphics[page=3,width=0.05\textwidth]{figures/forkingPath} &  $1 \times 3 \times 1 = 3$ 
 &$ 3/64 $ & $1/5$ & $\frac{3}{64}\frac{1}{5}$&$\frac{3}{3+8+9} = 0.15$\\[2pt]
  \includegraphics[page=4,width=0.05\textwidth]{figures/forkingPath} &  $2 \times 2 \times 2 = 8$ &$8/64$&$1/5$ &$\frac{8}{64}\frac{1}{5}$ & $\frac{8}{3+8+9}=0.40$\\[2pt]
  \includegraphics[page=5,width=0.05\textwidth]{figures/forkingPath} &  $3 \times 1 \times 3 = 9$ &$9/64$&$1/5$ &$\frac{9}{64}\frac{1}{5}$&$\frac{9}{3+8+9}=0.45$\\[2pt]
  \includegraphics[page=6,width=0.05\textwidth]{figures/forkingPath} &  $4 \times 0 \times 4 = 0$ &$0/64$&$1/5$ &$\frac{0}{64}\frac{1}{5}$&$\frac{0}{3+8+9}=0.00$\\[2pt] \cline{5-5} \\[-0.2cm]
 & & & & $\frac{3 + 8 + 9 }{64 \cdot 5} $ & 
\end{tabular}
\end{table}
\end{document}


