\documentclass[a4paper,10pt]{article}
\usepackage[utf8]{inputenc}
\input{../tex/encabezado.tex}
\input{../tex/tikzlibrarybayesnet.code.tex}

\usetikzlibrary{arrows}
\usetikzlibrary{shapes}
\usetikzlibrary{fit}
\usetikzlibrary{chains}

%\usetikzlibrary{shapes.geometric,positioning}


\title{ Ergodicity Breaking}

\author{Gustavo Landfried}
\affil{\small Universidad de Buenos Aires. Facultad de Ciencias Exactas y Naturales. Departamento de Computaci\'on. Buenos Aires, Argentina}
\affil[]{Correspondencia: \url{gustavolandfried@gmail.com}}



\begin{document}


\maketitle

Dada una matriz de pagos,

\begin{equation}
 v = \bordermatrix{ & P & N  \cr
      T & v(T,P) & v(T,N)  \cr
      F & v(F,P) & v(F,N)  \cr} 
\end{equation}

y una matriz de percepci\'on parametrizada

\begin{equation}
 p_{\lambda} = \bordermatrix{ & P & N  \cr
      T & p_{\lambda}(T,P) & p_{\lambda}(T,N)  \cr
      F & p_{\lambda}(F,P) & p_{\lambda}(F,N)  \cr} 
\end{equation}

seg\'un la teor\'ia de decisi\'on~\cite{gardner2019-optimalityDecisionTheory} el criterio de decisi\'on \'optimo es el que maximiza la utilidad esperada

\begin{equation}
 \lambda_{opt} = \underset{\lambda}{\text{arg max}} \text{utilidadEsperada}(V,p_{\lambda})
\end{equation}

donde 
\begin{equation*}
\text{utilidadEsperada}(V,p_{\lambda}) =  p_{\lambda}(T,P)\cdot v(T,P) + p_{\lambda}(T,N)\cdot v(T,N) + p_{\lambda}(F,P)\cdot v(F,P) + p_{\lambda}(F,N)\cdot v(F,N) 
\end{equation*}

Veamos un ejemplo.
Supongamos que tenemos un test con las siguientes caracter\'isticas.

\begin{figure}[H]
\centering
  \includegraphics[page=1,width=0.48\textwidth]{figures/ergodicity_breaking.pdf}
  \caption{}
  \label{}
\end{figure}

Y sabemos que los casos verdaderos y falsos ocurren con misma probabilidades.
Si quisieramos maximizar la cantidad de respuestas correctas, el criterio \'optimo estar\'ia ubicado en el medio de las dos distribuciones, $\lambda = 1$.
En este ejemplo, sin embargo, lo \'unico que no interesa es maximizar la probabilidad $p(T|P)$.
Por eso proponemos la siguiente matriz de pagos.

\begin{equation}
 v = \bordermatrix{ & P & N  \cr
      T & 1.00 & 0.00  \cr
      F & -0.99 & 0.00  \cr} 
\end{equation}

Solo nos importa lo que pasa cuando el test es positivo porque esos son los \'unico elementos que intervienen en el c\'alculo de $p(T|P)$

\begin{equation}
 p(T|P) = \frac{p(T,P)}{p(T,P)+p(F,P)}
\end{equation}

\begin{figure}[H]
\centering
  \includegraphics[page=2,width=0.48\textwidth]{figures/ergodicity_breaking.pdf}
  \caption{Criterio \'optimo y conservador (aversi\'on al riesgo).}
  \label{}
\end{figure}

\begin{figure}[H]
\centering
  \begin{subfigure}[t]{0.48\textwidth}
  \includegraphics[page=4,width=\textwidth]{figures/ergodicity_breaking.pdf}
  \caption{\'Optimo}
  \end{subfigure}
  \begin{subfigure}[t]{0.48\textwidth}
  \includegraphics[page=3,width=\textwidth]{figures/ergodicity_breaking.pdf}
  \caption{conservador}
  \end{subfigure}
  \caption{Probabilidades $p(P|T)$ (claro + oscuro) y $p(P|F)$ (oscuro). El n\'umero representa el likelihood ratio.}
  \label{}
\end{figure}



Lo que estamos calculando es el promedio de los estados del sistema.
Pero a nosotros lo que nos interesa es el resultado que como individuos efectivamente vamos a obtener el tiempo.
Cuando el promedio coincide con el equilibrio temporal de cualquier individuo, decimos que el sistema es ergódico, y podemos usar el primero para calcular el segundo.
Sin embargo, no es cierto en general~\cite{peters2019-ergodicityEconomics}.
Depende del proceso.

En evoluci\'on todos aprenden el hecho de que el crecimiento de una poblaci\'on es un proceso multiplicativo y ruidoso: una secuencia de probabilidades de supervivencia y reproducci\'on.
En los procesos multiplicativos los impactos f\'isicos de las perdidas son en general m\'as fuertes que las ganancias.
Con que haya un cero en la secuencia de generaciones, no hay posibilidad de recuperarse, estamos extintos.




{\scriptsize
\bibliographystyle{../biblio/plos2015}
\bibliography{../biblio/biblio_norUrl.bib}
}


\end{document}
